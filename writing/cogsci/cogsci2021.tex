% Template for Cogsci submission with R Markdown

% Stuff changed from original Markdown PLOS Template
\documentclass[10pt, letterpaper]{article}

\usepackage{cogsci}
\usepackage{pslatex}
\usepackage{float}
\usepackage{caption}

% amsmath package, useful for mathematical formulas
\usepackage{amsmath}

% amssymb package, useful for mathematical symbols
\usepackage{amssymb}

% hyperref package, useful for hyperlinks
\usepackage{hyperref}

% graphicx package, useful for including eps and pdf graphics
% include graphics with the command \includegraphics
\usepackage{graphicx}

% Sweave(-like)
\usepackage{fancyvrb}
\DefineVerbatimEnvironment{Sinput}{Verbatim}{fontshape=sl}
\DefineVerbatimEnvironment{Soutput}{Verbatim}{}
\DefineVerbatimEnvironment{Scode}{Verbatim}{fontshape=sl}
\newenvironment{Schunk}{}{}
\DefineVerbatimEnvironment{Code}{Verbatim}{}
\DefineVerbatimEnvironment{CodeInput}{Verbatim}{fontshape=sl}
\DefineVerbatimEnvironment{CodeOutput}{Verbatim}{}
\newenvironment{CodeChunk}{}{}

% cite package, to clean up citations in the main text. Do not remove.
\usepackage{apacite}

% KM added 1/4/18 to allow control of blind submission


\usepackage{color}

% Use doublespacing - comment out for single spacing
%\usepackage{setspace}
%\doublespacing


% % Text layout
% \topmargin 0.0cm
% \oddsidemargin 0.5cm
% \evensidemargin 0.5cm
% \textwidth 16cm
% \textheight 21cm

\title{Children's understanding of others' lexical knowledge}

\usepackage{booktabs}
\usepackage{longtable}
\usepackage{array}
\usepackage{multirow}
\usepackage{wrapfig}
\usepackage{float}
\usepackage{colortbl}
\usepackage{pdflscape}
\usepackage{tabu}
\usepackage{threeparttable}
\usepackage{threeparttablex}
\usepackage[normalem]{ulem}
\usepackage{makecell}
\usepackage{xcolor}



\newlength{\cslhangindent}
\setlength{\cslhangindent}{1.5em}
\newenvironment{CSLReferences}%
  {}%
  {\par}

\begin{document}

\maketitle

\begin{abstract}
Communicating effectively requires keeping track of what others know,
and what they do not know. As a result, young children's developing
ability to communicate effectively also requires them to learn to infer
and update what others are likely to know. We asked children ages 4-8
(\emph{n} = 62) to make specific predictions about whether a very young
child would know 15 familiar animal words. With minimal information
about the target child, children as young as 4 years old are able to
make item-level predictions, inferring the target child would be more
likely to know easy, early-acquired words (e.g., \emph{dog}) and less
likely to know harder, later-acquired words (e.g., \emph{lobster}). We
also discuss children's explanations for why words might be known or
unknown. In sum, this work suggests that even preschool age children are
making sophisticated inferences about what words another child might
know.

\textbf{Keywords:}
communication, metalinguistics, knowledge reasoning, cognitive
development
\end{abstract}

\hypertarget{introduction}{%
\section{Introduction}\label{introduction}}

Imagine visiting the zoo with your friend and their 2-year-old. As you
walk by the peacocks, you hear your friend say, ``Do you see those blue
birds?'' Immediately, you know that your friend is talking to their
child and not you. While ``peacock'' would be a good word choice for
talking to you, ``blue bird'' is a much better description for a child
who has never seen a peacock before. Even when referring to the same
object, we choose different words when talking to different people
because of what we think our conversational partners know and don't
know.

Adults track and adapt to their conversational partners' knowledge with
relative ease. For example, adults reduce the information they give when
re-telling a story to someone who has heard it before, but not when
telling the story to a new partner (Galati \& Brennan, 2010). Adults can
adapt even to partners who are quite different from them, as in the case
of parents and their children. Parents have accurate models of their
children's vocabularies, and use these models in spontaneous
communication, as in the ``blue bird'' example above {[}Fenson et al.
(2007); Leung, Tunkel, \& Yurovsky, in press; Masur (1997){]}. While
parents' models of their children's vocabulary are undoubtedly shaped by
extensive individual interactions, they likely also rely on general
metalinguistic knowledge--for instance, that shorter words are generally
simpler than longer words and thus more likely to be in children's
vocabularies.

Even in the absence of experience with a specific interlocutor, adults
can rely on other metalinguistic knowledge. In a large-scale study,
Kuperman, Stadthagen-Gonzalez, and Brysbaert (2012) asked adult
participants to report the age at which they understood a given word and
obtained judgments for 30,000 English words. These judgments were then
directly compared with data on the typical age that a given word is
actually learned (also called its age of acquisition, hereafter referred
to as AoA). Adults typically overestimate the absolute age at which they
learned a given word; however, the estimated order in which words are
acquired is intact (Kuperman, et al., 2012). Adults are thus able to
make graded and surprisingly accurate relative estimates of when a word
was learned.

Young children show an impressive ability to track other people's
knowledge across a wide range of situations. Reasoning about another
person's specific lexical knowledge may prove difficult for young
children as they also show consistent errors in reasoning about other's
knowledge, commonly over-attributing knowledge (e.g.~Gopnik \&
Astington, 1988; Taylor, Esbensen, \& Bennett, 1994). The bias to
over-attribute knowledge is particularly pronounced when the child
themselves knows the piece of information (Birch \& Bloom, 2003). After
seeing inside a toy, 3-year-old children often attributed knowledge of
what was inside the toy to a puppet that had never played with the toy
(Birch \& Bloom, 2003). Such curse-of-knowledge biases could severely
hinder children's ability to reason about the lexical knowledge of
another child.

{[}better bridge to our study{]}

In this study, we ask whether children have accurate estimates of other
children's knowledge. Specifically, we are interested in whether
children are sensitive to the specific vocabulary knowledge of a younger
child and are able to make item-wise predictions. Such predictions are
crucial to communicate effectively with various interlocutors and
account for varying knowledge and perspectives. By age 5, children
richly structure their language based on a listener's knowledge, for
example offering more general information to ignorant listeners (Baer \&
Friedman, 2018). However, such studies typically strongly and repeatedly
emphasize an interlocutors' knowledge state, to test for children's
ability to adapt their communication while controlling for difficulties
in knowledge reasoning. Children's performance in other communication
tasks (e.g.~Krauss \& Glucksberg, 1977) might be hindered by difficulty
specific knowledge predictions.

To our knowledge, only one study has asked children to give AoA
estimates for English words (Walley \& Metsala, 1992). In their study,
Walley and Metsala (1992) used a broad set of words acquired over a
large range of ages (from table to valet) to investigate young
children's general metalinguistic knowledge. As young as age 5, children
generated AoA estimates that were similar to adults'. Our study builds
on Walley and Metsala's (1992) work to collect more sensitive AoA
estimates in a single domain (animal words). To probe the specificity
and sensitivity of children's AoA estimates, the animal words we use in
this study are generally acquired within a narrow age range of 2 to 2.5,
based on parent reports of children's vocabulary (Wordbank). Our study
also differs from Walley and Metsala's (1992) in a crucial way--rather
than asking children when they themselves learned a word, we ask them to
estimate the vocabulary knowledge of another fictional child. This
framing also allowed us to not ask children the age at which they
learned a word, but instead their certainty about the other child's
knowledge. Lastly, our study also probes whether children as young as 4
might show this capacity to reason about another child's lexical
knowledge.

In the current study, children ages 4-8 were introduced to a younger
fictional child, and asked to make judgments about this fictional
child's knowledge of various animal words. We expected that overall,
children's judgments would recover the ordinal shape of age of
acquisition data for these items. That is, children would infer that the
child is most likely to know early acquired words, yielding a negative
correlation between their judgments of lexical knowledge and adult AoA
estimates. We expect developmental change in children's sensitivity to
Sam's vocabulary knowledge, with older children's judgments recovering
word-level AoA data more closely.

\hypertarget{method}{%
\section{Method}\label{method}}

\hypertarget{stimuli}{%
\subsubsection{Stimuli}\label{stimuli}}

Our stimuli consisted of 15 words drawn from a single domain (animal
words), along with corresponding images of each animal. We pulled all
animal images (n = 45) from a normed image set (Rossion \& Pourtois,
2004; recoloring of Snodgrass \& Vanderwart, 1980). To ensure our
stimuli set spanned a large AoA range, we ranked the animal words from
earliest to latest AoA, using data from Kuperman et al.~(2012), and
split the words into five bins. In order to select animal images that
are recognizable and typically identified by a single name, we chose the
three animals from each AoA quintile with the highest naming agreement
according to a naming task with children (Cycowicz et al., 1997). Our
final stimuli consisted of these 15 items, ordered here by estimated
AoA: dog, duck, cat, pig, fish, turtle, zebra, elephant, snake, penguin,
gorilla, owl, raccoon, leopard, and lobster. While adult AoA estimates
for these words range from 2.5 to 7.5 years old, all of these animal
words are generally acquired by age 3 according to parent reports of
children's vocabulary knowledge (Wordbank?). Because the youngest
children in our study are 4 years old, we expect all participants to
know these animal words.

\hypertarget{participants}{%
\subsubsection{Participants}\label{participants}}

We pre-registered a planned sample of 60 children ages 4-8, with 12
children recruited for each year-wise age group. Due to overrecruitment,
our final sample included 62 children (12 4-year-olds, 13 5-year-olds,
13 6-year-olds, 12 7-year-olds, 12 8-year-olds). All analyses hold when
looking only at the 60 children run first chronologically. Based on our
pre-registered exclusion criterion, children who failed to answer all of
the questions were excluded and replaced (an additional 7 children).
Families were recruited online, primarily through a US University
database of families who have expressed interested in doing research or
previously participated. Children completed this study over Zoom,
interacting with a live experimenter who navigated a slide-style,
animated Qualtrics survey.

A separate sample of 30 adults were recruited via Amazon Mecchanical
Turk. The adult sample provides a simple test that our task elicits
robust inferences about the target child's lexical knowledge, and that
these inferences correspond to extant AoA data. The adult participants
completed the same task using Qualtrics, with minor modifications
detailed below.

\hypertarget{procedure}{%
\subsection{Procedure}\label{procedure}}

\begin{CodeChunk}
\begin{figure}[tb]

{\centering \includegraphics{figs/task-method-1} 

}

\caption[Schematic showing the general structure of an example trial]{Schematic showing the general structure of an example trial. The experimenter labels the animal, and asks the child “Do you think Sam knows that this is called an elephant?” Based on their response, children are then asked to provide a confidence judgment on a 3-point scale (little sure, medium sure, very sure).}\label{fig:task-method}
\end{figure}
\end{CodeChunk}

\emph{Introduction.} Children were shown a picture of a child named
``Sam'' (seen in Figure \ref{fig:task-method}). Children were anchored
to Sam's knowledge of various familiar skills, specifically some skills
that Sam has acquired (e.g., coloring), and some that Sam has not (e.g.,
reading). Children are then specifically anchored to Sam's possible word
knowledge in an unrelated domain-- given an example of one word Sam
knows (car), and one word that Sam doesn't know (piano). This
introduction is intended to familiarize the children with Sam, roughly
anchor them to Sam's knowledge and age, and to ensure that children
understand there are things Sam doesn't know yet (even things children
themselves likely know, such as how to read).

\emph{General trial structure.} At each trial, children were shown a
drawing of a familiar object or animal (drawings taken from Rossion \&
Pourtois, 2004, which is a recoloring of Snodgrass \& Vanderwart, 1980).
The experimenter labelled the object for the child (e.g., ``Look, it's a
{[}ball{]}! Do you think Sam knows that this is called a {[}ball{]}? Yes
or no?''). Based on their response, children are then asked a follow-up
question: ``How sure are you that Sam {[}knows/doesn't know{]} that this
is called a {[}ball{]}-- a little sure, medium sure, or very sure?'' All
questions were presented with accompanying pictures of thumbs
{[}up/down{]} of varying size (see Figure \ref{fig:task-method}).
Children as young as 3 are able to engage in uncertainty monitoring and
report confidence, although these skills do develop in the preschool
years (Lyons \& Ghetti, 2011). Children's responses to these two items
were recoded onto a 1-6 scale from 1-very sure Sam doesn't know to
6-very sure Sam knows. All trials followed this general structure. The
experimenter provided no evaluative feedback on any trials, but did
offer consistent neutral feedback (e.g., repeating the child's answer or
saying ``Okay!''). When a child failed to respond within about 5 seconds
or offered a non-canonical response (e.g., saying ``Maybe''), the
experimenter acknowledged the child's answer and then repeated the
question with the possible responses. If a child did not answer after
the question was repeated, the experimenter moved on and marked the
trial as no response.

\emph{Familiarization trials.} Children first completed two non-animal
familiarization trials, one of an early-acquired word (ball) and one of
a late-acquired word (artichoke). These trials followed the trial
structure described above and were intended to help familiarize children
with the structure of the questions and scales. These trials were always
asked first and in a fixed order.

\emph{Animal trials.} Children were then shown 15 trials of the same
form (see example trial in Figure \ref{fig:task-method}). For the 15
animal trials, trial order was randomized across participants to control
for any potential order effects in children's responses.

\emph{Explanation.} After completing the final animal trial, children
were asked an open-ended explanation question about their final
judgement (e.g., ``Why do you think Sam {[}knows/doesn't know{]} that
this is called {[}an elephant{]}?''). Because the trial order was
randomized, the explanations concerned different animal words across
participants.

\emph{Final check questions.} Children were asked two questions about
Sam's skill knowledge, one early acquired skill (going up and down
stairs) and one very late acquired skill (driving a car). These
questions again followed the general trial structure described above.
The skill knowledge items were included as an additional check that
children at all ages were able to use the scale appropriately, in case
young children failed to differentiate animal words based on AoA.
Lastly, children were asked to report how old they thought Sam was. This
question was intended to assess another aspect of children's belief
about Sam. Sam's photo and skill knowledge were intended to indicate
toddlerhood.

\hypertarget{adult-procedure.}{%
\paragraph{Adult procedure.}\label{adult-procedure.}}

The adult participants completed a minimally adapted version of the same
task online via Qualtrics. Unlike children, adults were simply presented
with the full 6-point scale (1 - \emph{very sure Sam doesn't know} to 6
- \emph{very sure Sa does know}). Additionally, the task was
administered asychronously, so adult participants did not interact with
an experimenter or recieve any feedback during the task. Otherwise, the
adult task was identical to the child task descirbed above.

\begin{CodeChunk}
\begin{figure}[tb]
\includegraphics{figs/overall-1} \caption[Comparing adult AoA estimates (in years, taken from Kuperman et al., 2012) and children’s judgments on our 6-point scale (1 = very sure Sam doesn’t know]{Comparing adult AoA estimates (in years, taken from Kuperman et al., 2012) and children’s judgments on our 6-point scale (1 = very sure Sam doesn’t know; 6 = very sure Sam knows). The black lines show 95\% confidence intervals for each item. The shaded region shows one standard deviation based on a linear regression estimated from the raw data.}\label{fig:overall}
\end{figure}
\end{CodeChunk}

\hypertarget{results}{%
\section{Results}\label{results}}

Our primary analyses compare knowledge judgments on our 6-point scale to
AoA judgements from adults (taken from Kuperman et al., 2012). Data were
analyzed using pre-registered mixed effects model predicting children's
judgments from adult AoA estimates (Kuperman et al., 2012), including
random effects for participant and word. Using the lme4 package in R
(\textbf{citation}), our model syntax was
\texttt{judgment\ \textasciitilde{}\ aoa\ +\ (1\ \textbar{}\ participant)\ +\ (1\textbar{}word)}.

First, analyzing adults responses on our task, we the predicted negative
effect of AoA on adult's judgements of the target child's knowledge (see
(Figure \ref{fig:development}), \(\beta =\) -0.63, \(t =\) -8.71, \(p\)
\textless{} .01). This provides a simple sanity check that our task is
eliciting reliable predictions from adults, that match predictions from
AoA estimation tasks (e.g., Kuperman et al., 2012).

\begin{CodeChunk}
\begin{figure*}[tb]
\includegraphics{figs/development-1} \caption[Children’s judgements across development]{Children’s judgements across development. Comparing adult AoA estimates (in years, taken from Kuperman et al., 2012) and children’s judgments, split by age in years.}\label{fig:development}
\end{figure*}
\end{CodeChunk}

Do children's judgments about a child's vocabulary knowledge also
reflect a sensitivity to which words are learned later? To answer this
quetion looking at children's responses overall, we ran the model with
no age term and see a significant negative effect of AoA on children's
judgements (\(\beta =\) -0.65, \(t =\) -8.29, \(p\) \textless{} .001).
That is, overall, children judged that the target child would be most
likely to know an early acquired word (e.g., dog) and least likely to
know a late acquired word (e.g., lobster, see (Figure
\ref{fig:overall})).

To test for developmental changes in children's responses, we used the
same mixed effects model but included an effect of age and an
interaction between AoA and age. Our model syntax was
\texttt{judgment\ \textasciitilde{}\ aoa\ *\ age\ +\ (1\ \textbar{}\ participant)\ +\ (1\textbar{}word)}.
We expected that older children's judgments would more closely reflect
word-level AoA data, yielding a significant interaction between AoA and
child's age. That is, when plotting children's judgments against adult
AoA estimates, older children would show steeper negative slopes than
younger children. Again, our model shows the same main effect of aoa
that we saw in the overall model (\(\beta =\) -0.65, \(t =\) -8.29,
\(p\) \textless{} .001). We also see a positive main effect of
children's age on their ratings (\(\beta =\) 0.55, \(t =\) 3.68, \(p\)
\textless{} .001). Crucially, we see our expected interaction between
child's age and adult's estimated AoA (\(\beta =\) -0.14, \(t =\) -5.1,
\(p\) \textless{} .001), suggesting that children's judgements are
becoming more adult-like in this age range (Figure
\ref{fig:development}).

To test the robustness of this intuition at each age, we ran the above
model separately for each year-wise age group. While we see evidence of
developmental change above, this additional analysis helps us understand
if even young children are showing this intuition. We found a
significant negative effect of AoA on children's judgments at all age
groups (with the smallest effect in 4-year-olds: \(\beta =\) -0.33,
\(t =\) -3.01, \(p =\) .01). That is, even 4-year-old children judged
that late-acquired animal words were less likely to be known by the
target child.

\hypertarget{explanations}{%
\subsubsection{Explanations}\label{explanations}}

As a secondary analysis, we were also interested in the reasons young
children gave for why the target child would or would not know a given
word. While children sometimes offered spontaneous explanations
throughout the study, this analysis focuses on the explanation elicited
after the final animal trial. Based on preliminary discussions between
the authors, the explanations were divided into 5 non-mutually-exclusive
categories: \emph{Language}, \emph{Experience}, \emph{Location},
\emph{Age}, \emph{Unsure}, and \emph{Other}.

\emph{Language} reflects explanations that explicitly appealed to
language properties (e.g., ``becuase it's a hard word'').
\emph{Experience} reflects explanations that appeal to the child's
real-world experience with the referent (e.g., ``because they might have
it for a pet''). \emph{Location} reflects explanations that specifically
reference a particular place the animal is associated with (e.g.,
``because it lives in the water''). \emph{Age} reflects explanations
that reference a particular age or general age-group (``lots of babies
don't know about them''). Any child that failed to answer the
explanation question or expressed ignorance was coded as giving an
explanation of \emph{Unsure}. An explanation that didn't fall into any
of the above category was coded as \emph{Other}. Note that coding was
not mutually-exclusive, so that explanations could be coded as including
multiple categories. See Table \ref{tab:explanations_table} for examples
of each coding category.

\begin{table*}[tb]
\centering
\begin{tabular}{ll}
  \hline
Category & Example Utterance \\ 
  \hline
Language & Because it was a very long word. \\ 
  Experience & Because maybe he has a dog. \\ 
  Location & Because penguins live in the artic and it's too cold for little kids so that's why you should have 130 jackets... \\ 
  Age & Because I think I knew that when I was around 3, I knew what a pig was. \\ 
  Unsure & I don't know. \\ 
  Other & Because it had a longer beak than a bird. \\ 
   \hline
\end{tabular}
\caption{Example explanations from child participants for each of the five categories used for coding.} 
\label{tab:explanations_table}
\end{table*}

\hypertarget{discussion}{%
\section{Discussion}\label{discussion}}

Our ability to track other people's knowledge is crucial for successful
communication. Young children are capable of inferring others' general
knowledge states, but do they make accurate judgments about another
person's specific knowledge. We asked 4- to 8-year-old children to
estimate a fictional child's knowledge of animal words, and found that
children as young as 4 are sensitive to a younger child's lexical
knowledge. Children across age groups made judgments similar to those of
adults, with older children recovering more adult-like patterns.

Our findings indicate that young children have surprising metalinguistic
knowledge, and can use that knowledge to make highly-specific inferences
about other people's knowledge. The animal words used in our study are
generally learned within a 6-month period, yet young children could
still distinguish early-acquired words from late-acquired words in this
set. Our study also builds upon the extant literature on children's
inferences about other people's knowledge to show that children infer
others' specific, lexical knowledge. When given fairly minimal
information about a fictional child, children readily make estimates
about that child's lexical knowledge.

How are children in our study making estimates about other people's
knowledge? One limitation of the current study is that it leaves the
mechanisms underlying such estimates unclear. Children's own
explanations suggest that they use various cues to make their estimates,
mostly appealing to {[}age/language/experience?{]}. When making
judgments about someone else's vocabulary knowledge, children use
information about the person, as well as general knowledge about the
word and its referent to inform their estimations. Future should could
more directly probe the features underlying this inference-- to see if
children are relying on their own uncertainty, word length (and other
linguistic cues), features of the referent itself, or still other
features.

The current work lays the foundation for future research on how children
leverage their knowledge of other people to communicate successfully.
Young children struggle in a variety of communicative tasks (e.g.~Krauss
\& Glucksberg, 1977), and the current work can begin to map out whether
such difficulties stem from tracking an interlocutor's knowledge, or may
stem from problems using that information to adjust language production.
By at least age 5, children selectively talk about general or specific
characteristics of an object based on their partners' knowledge state,
when the knowledge state is salient and explicit for each item (Baer \&
Friedman, 2018). Based on our findings that children can reason about
others' specific knowledge, we can ask whether children's adaptations
extend to the level of lexical knowledge-- Do children adjust the way
they talk about a referent based on their beliefs about a partners'
knowledge of that word?

\vspace{1em} \fbox{\parbox[b][][c]{7.3cm}{\centering Stimuli, data, and analysis code available after deanonymization.}}

\hypertarget{references}{%
\section{References}\label{references}}

\setlength{\parindent}{-0.1in} 
\setlength{\leftskip}{0.125in}

\noindent

\hypertarget{refs}{}
\begin{CSLReferences}{1}{0}
\leavevmode\hypertarget{ref-fenson2007}{}%
Fenson, L., Marchman, V. A., Thal, D. J., Dale, P. S., Reznick, J. S.,
\& others. (2007). \emph{MacArthur-bates communicative development
inventories: User's guide and technical manual}. Baltimore, MD: Brookes.

\leavevmode\hypertarget{ref-masur1997}{}%
Masur, E. F. (1997). {Maternal labelling of novel and familiar objects:
implications for children's development of lexical constraints}.
\emph{Journal of Child Language}, \emph{24}, 427--439.

\end{CSLReferences}

\bibliographystyle{apacite}


\end{document}
